\documentclass{article}
% Change "article" to "report" to get rid of page number on title page
% \usepackage{amsmath,amsfonts,amsthm,amssymb}
\usepackage{setspace}
\usepackage{fancyhdr}
\usepackage{lastpage}
\usepackage{extramarks}
\usepackage{textcomp}
\usepackage{amsmath}
\usepackage{lstcustom}
\usepackage{enumitem}
\usepackage{multicol}
\usepackage{url}

\usepackage{tikz}

\usepackage{totcount} % For the total points

% In case you need to adjust margins:
\topmargin=-0.45in
\evensidemargin=0in
\oddsidemargin=0in
\textwidth=6.5in
\textheight=9.0in
\headsep=0.25in

% Commands for points
\newtotcounter{points}
\newcommand*{\totalpoints}{\thepoints}
\newcommand{\pts}[1]{\addtocounter{points}{#1}(#1pt)}

\newtotcounter{ecpoints}
\newcommand*{\totalecpoints}{\theecpoints}
\newcommand{\ecpts}[1]{\addtocounter{ecpoints}{#1}(#1pt)}

% Exercise Specific Information
\newcommand{\hmwkTitle}{\underline{HW 01}}
\newcommand{\hmwkDueDate}{Due 2016-07-10 8:00 a.m.}
\newcommand{\hmwkClass}{COS 101}
\newcommand{\hmwkPoints}{ \ref{lastQuestion} questions; \protect\total{points} points + \protect{\total{ecpoints}} ec; \pageref{LastPage} pgs.}

\newcommand{\wordspace}[1][1]{\underline{\hspace{#1in}}}

% Setup the header and footer
\pagestyle{fancy}
\lhead{\hmwkClass\ \hmwkTitle} 
\chead{\hmwkPoints}
\rhead{\hmwkDueDate}
\lfoot{\lastxmark}  
\cfoot{}            
\rfoot{Page\ \thepage\ of\ \pageref{LastPage}}
\renewcommand\headrulewidth{0.4pt} 
\renewcommand\footrulewidth{0.4pt}

% Whitespace symbol using the beramono font.
\newcommand{\ws}{\texttt{\textvisiblespace}}

\newcommand{\truefalse}[1]{
  \item \underline{\hspace{2cm}} \pts{2} #1
}

\newcommand{\anspace}{
  \underline{\hspace*{1in}}
}

\newcommand{\tspace}{\vspace{4cm}}

% Setup the code listings look
\lstset{
  language=C++,
  style=eclipse,
  showspaces=false, 
  numbers=left,
  frame=none
}


\newcommand{\op}[1]{\lstinline$#1$}

\begin{document}
\begin{spacing}{1.1}

\begin{enumerate}[leftmargin=*]
\item \pts{1} \textbf{Name:} \hrulefill % \underline{\hspace{4in}}

Both the electronic and paper portion of this assignment are due at the
beginning of class.  Assignments after 8am will be considered late.
Ensuring that the electronic portion of the assignment is submitted
on-time and in the correct format is critical.  Make sure to leave
yourself plenty of time to submit the project.

\item \pts{4} Write the MD5 hash of your submitted code
  here. \hrulefill % \underline{\hspace*{3in}}\\

~

\item \pts{4} How many iterations does the following loop make? \anspace

\begin{lstlisting} 
int count = 1;
while (count < 30) {
   count = count * 2;
}
\end{lstlisting}

\item \pts{4} How many iterations does the following loop make? \anspace

\begin{lstlisting}
int count = 15;
while (count < 30) {
   count = count * 3;
}
\end{lstlisting}

\item \pts{4} How many iterations does the following loop make? \anspace

\begin{lstlisting}
int count = 1;
while (count < n) {
   count = count * 2;
}
\end{lstlisting}

\item \pts{4} How many iterations does the following loop make? \anspace

\begin{lstlisting}
int count = 15;
while (count < n) {
   count = count * 3;
}
\end{lstlisting}

\item \pts{24} For each of the following code snippets, determine how many
  stars are displayed for $n=5,10,20$.  Use the Big $O$ notation to
  estimate the time complexity.

\begin{enumerate}

\item ~

\vspace{-\baselineskip}
\begin{lstlisting}[xleftmargin=.25in]
for (int i = 0; i < n; i++) {
   cout << '*';
}
\end{lstlisting} 

\tspace

\item ~

\vspace{-\baselineskip}
\begin{lstlisting}[xleftmargin=.25in]
for (int i = 0; i < n; i++) {
   for (int j = 0; j < n; j++) {
      cout << '*';
   }
}
\end{lstlisting} 

\tspace

\item ~

\vspace{-\baselineskip}
\begin{lstlisting}[numbers=none]
for (int k = 0; k < n; k++) {
   for (int i = 0; i < n; i++) {
      for (int j = 0; j < n; j++) {
         cout << '*';
      }
   }
}
\end{lstlisting} 

\tspace

\item ~

\vspace{-\baselineskip}
\begin{lstlisting}[numbers=none]
for (int k = 0; k < 10; k++) {
   for (int i = 0; i < n; i++) {
      for (int j = 0; j < n; j++) {
         cout << '*';
      }
   }
}
\end{lstlisting} 

\tspace

\end{enumerate}

\newpage

\item \pts{8} Design an $O(n)$ time algorithm for computing the sum of
  integers from $n_1$ to $n_2$ where $0 \le n_1 \le n_2$.

\tspace

\item \pts{7} Is it possible to improve the performance of the algorithm from
  the previous question so that it is $O(1)$ (a constant time
  algorithm)? If it is possible, provide the algorithm; if not,
  explain why it is not possible to create such an algorithm.

\tspace

\end{enumerate}

\subsection*{Electronic Submission}

\lstset{
  breaklines=false
}

\begin{enumerate}[leftmargin=*, resume]

\item\label{lastQuestion} \pts{80} You will write a program that will
  aid in evaluating the performance of the sorting routines we have
  discussed in class.  Specifically, your program should perform tests
  on the \lstinline$bubbleSort$, \lstinline$insertionSort$,
  \lstinline$mergeSort$, and \lstinline$selectionSort$ functions
  defined in \lstinline$sort.cpp$ that is posted to the course
  documents
  page\footnote{\url{https://raw.githubusercontent.com/wiki/markroyer/latex-homework-template/homework-examples.tgz}}. Modify
  the functions so that they return a \lstinline$long long$ value that
  represents the amount of operations required to complete the sort.

Typing \textbf{make} at the root of your project submission will
create an executable file called \textbf{analyze}. Your program will
make use of commandline arguments to determine which files to
load, what sorting routine to execute, and the number of files that
will be loaded.  You will make use of the files 

\textbf{asc1.txt, \dots , asc10.txt},

\textbf{desc1.txt, \dots , desc10.txt}, and

\textbf{random1.txt, \dots , random10.txt}

that are posted to the course documents page. The \textbf{asc},
\textbf{desc}, and \textbf{random} files contain dictionary words in
ascending, descending, and random order respectively. The program
should be invoked in the following manner:

\begin{verbatim}
analyze SORT FILENAMEPREFIX NUM
\end{verbatim}

The program arguments are defined as follows:

\begin{description}

\item[SORT] \hfill \\ The name of one of the sorting routines mentioned above.  Valid
options are \lstinline$bubbleSort$, \lstinline$insertionSort$,
\lstinline$mergeSort$, and \lstinline$selectionSort$.

\item[FILENAMEPREFIX] \hfill \\ The name of the file to be loaded without the
  number or \texttt{.txt} extension.  For this assignment, we will use \lstinline$asc$,
  \lstinline$desc$, and \lstinline$random$.

\item[NUM] \hfill \\ The number of files to be read.

\end{description}

The output of the program is a single file named
\textbf{SORT-FILENAMEPREFIX.data} that contains two columns of data
separated by a single tab (\lstinline$"\t"$).  The first column is the
number of words in the file, and the second column is the amount of
operations it took for the function to sort the data.  The output file
may have lines containing descriptive information as long as those
lines begin with a \texttt{\#} symbol.

\subsubsection*{Example}

Suppose that the program was run with the options as shown below:

\begin{verbatim}
analyze selectionSort asc 10
\end{verbatim}

This would generate the output file
\textbf{\texttt{selectionSort-asc.data}}.  The contents of the file
would have ten lines of output (one for each file) and look similar to
the following:

\begin{lstlisting}[language=java,numbers=none]
# Result of running selectionSort on asc1-10
# n	ops
10	100
100	10000
1000	1000000
etc ...
\end{lstlisting}

For each sorting routine, use the generated output file from your
program to graph the results versus the expected time
complexity for the chosen sorting algorithm. Put these results and a
brief discussion of why or why not your results match the theoretical
ones in a single pdf file named \textbf{results.pdf}.

\textbf{Note:} As with previous assignments, no other output should be
generated by the program.  You will lose points if the specification
is not strictly adhered to.

\textbf{Submission:} Submit a single tgz file to
\url{http://yoursite.com/u/} using the following
naming convention.

\begin{verbatim}
lastname-hwNN-SNUM.tgz
\end{verbatim}

You should replace \emph{lastname} with your last name, \emph{NN} with
the assignment number (eg. 05), and \emph{SNUM} with a four digit
number specifying the submission version.  Extracting the contents of
the tarball should create a single folder containing your project
following the naming convention above without the extension.  The
folder should contain the following files:

\begin{description}
  \item[README] A file that describes the submission content and how
    to build and run your program.
  \item[Makefile] The script for building your project.
  \item[Doxyfile] Doxygen input file for creating documentation.
  \item[sort.cpp] Source file containing the main function, the
    sorting functions, and any other additional functions that may be
    needed.
  \item[results.pdf] 
\end{description}

\textbf{Important:} Make sure that your submission does not include
generated files such as object files and doxygen output
documentation. You may lose points if this is not strictly followed.

\textbf{Questions:} For clarification, please post additional questions to the
newsgroup.

\end{enumerate}

~\\
~\\

\noindent
\textbf{Extra Credit*} \ecpts{10} In the textbook the author claims
that $O(\log n) = O(\log_2 n) = O(log_a n)$.  Does the same hold for
$\Theta$? That is, can we say that for all positive real numbers $a$
and $b$ where $a \ne 1, b \ne 1$ does $\Theta(\log_a n) = \Theta(\log_b n)$? If so, prove that the
equality holds, otherwise show why it does not.

\vspace{12cm}

\noindent
\textbf{Extra Credit*} \ecpts{30} Modify the \textbf{analyze} program
to support \textbf{externalSort}. Perform the same analysis of the
sorting routine using the ascending, descending, and random input
files from the homework and summarize the results.  Append these
results to the end of your \textbf{results.pdf} file.
\end{spacing}


\end{document}


